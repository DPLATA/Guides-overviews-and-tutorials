%%%%%%%%%%%%%%%%%%%%%%%%%%%%%%%%%%%%%%%%%%%%%%%%
% Vim Cheat Sheet
%
% Created and edited by DGPC
% 
% Template obtained from: https://www.overleaf.com/articles/130-cheat-sheet/ntwtkmpxmgrp
%
%
%%%%%%%%%%%%%%%%%%%%%%%%%%%%%%%%%%%%%%%%%%%%%%%%

%----------------------------------------------------------------
%	PACKAGES AND OTHER DOCUMENT CONFIGURATIONS
%----------------------------------------------------------------

\documentclass{article}
\usepackage[landscape]{geometry}
\usepackage{url}
\usepackage{multicol}
\usepackage{amsmath}
\usepackage{esint}
\usepackage{amsfonts}
\usepackage{tikz}
\usetikzlibrary{decorations.pathmorphing}
\usepackage{amsmath,amssymb}

\usepackage{colortbl}
\usepackage{xcolor}
\usepackage{mathtools}
\usepackage{amsmath,amssymb}
\usepackage{enumitem}
\makeatletter

\newcommand*\bigcdot{\mathpalette\bigcdot@{.5}}
\newcommand*\bigcdot@[2]{\mathbin{\vcenter{\hbox{\scalebox{#2}{$\m@th#1\bullet$}}}}}
\makeatother

\title{Vim editor Cheat Sheet}
\usepackage[brazilian]{babel}
\usepackage[utf8]{inputenc}

\advance\topmargin-.8in
\advance\textheight3in
\advance\textwidth3in
\advance\oddsidemargin-1.5in
\advance\evensidemargin-1.5in
\parindent0pt
\parskip2pt
\newcommand{\hr}{\centerline{\rule{3.5in}{1pt}}}
%\colorbox[HTML]{e4e4e4}{\makebox[\textwidth-2\fboxsep][l]{texto}
\begin{document}

\begin{center}{\huge{\textbf{Vim editor Cheat Sheet}}}\\
\end{center}
\begin{multicols*}{3}

\tikzstyle{mybox} = [draw=black, fill=white, very thick,
    rectangle, rounded corners, inner sep=10pt, inner ysep=10pt]
\tikzstyle{fancytitle} =[fill=black, text=white, font=\bfseries]

%------------ Exiting  ---------------
\begin{tikzpicture}
\node [mybox] (box){%
    \begin{minipage}{0.3\textwidth}
    :q quit file
    \\
    :wq write bufferstream and quit file
    \\
    :q! quit file and abandon changes
    \\
    :qa quit all files
    \\
    :qa! quit all files and abandon changes
    \end{minipage}
};
%------------ Exiting Header ---------------------
\node[fancytitle, right=10pt] at (box.north west) {Exiting};
\end{tikzpicture}

%------------ Basic Navigation ---------------
\begin{tikzpicture}
\node [mybox] (box){%
    \begin{minipage}{0.3\textwidth}
    h - left
    \\
    j - down
    \\
    k - up
    \\
    l - right
    \end{minipage}
};
%------------ Basic Navigation Header ---------------------
\node[fancytitle, right=10pt] at (box.north west) {Basic Navigation};
\end{tikzpicture}

%------------ Template Content ---------------
\begin{tikzpicture}
\node [mybox] (box){%
    \begin{minipage}{0.3\textwidth}
    \begin{align*}
    \end{align*}
    \end{minipage}
};
%------------ Template Header ---------------------
\node[fancytitle, right=10pt] at (box.north west) {Title template};
\end{tikzpicture}
%------------ Template Content ---------------------
\begin{tikzpicture}
\node [mybox] (box){%
    \begin{minipage}{0.3\textwidth}
    	\begin{align*}
            \\
            \\
            \\
            \\
            \\
            \\
            \\
            \\
    	\end{align*}
    \end{minipage}
};
%------------ Template Header ---------------------
\node[fancytitle, right=10pt] at (box.north west) {Title template};
\end{tikzpicture}

%------------ Navigation Content ---------------
\begin{tikzpicture}
\node [mybox] (box){%
    \begin{minipage}{0.3\textwidth}
    \small{
    	\begin{tabular}{lp{4cm} l}
		\textit{Words:} & forward / backwards
        \\ & w / b
        \\
        \\ \hline

		\textit{Document:} & First line
        \\ & gg
        \\ & Last line
        \\ & G
        \\
        \\ \hline
		\textit{Window:} & Center current line
        \\ & zz
        \\ & Current line to top
        \\ & zt
        \\ & Cursor to top of the screen
        \\ & H
        \\ & Cursor to middle of the screen
        \\ & M
        \\ & Cursor to bottom of the screen
        \\ & L
	\end{tabular}}
    \end{minipage}
};
%------------ Navigating Header ---------------------
\node[fancytitle, right=10pt] at (box.north west) {Navigating};
\end{tikzpicture}

%------------ Series Solution Content ---------------
\begin{tikzpicture}
\node [mybox] (box){%
    \begin{minipage}{0.3\textwidth}
    $y'' + p(x)y' + q(x)y = 0$ \\
    Useful when $p(x), q(x)$ not constant \\
    Guess $y = \sum_{n=0}^{\infty}a_n(x-x_0)^n$
    \small{
    	\begin{tabular}{lp{4cm} l}
        \hline
        $e^x$ & $\sum_{n=0}^{\infty} x^n/{n!}$ \\ \hline
        $\sin x$ & $\sum_{n=0}^{\infty} \frac{(-1)^n}{(2n+1)!}x^{2n+1}$ \\ \hline
        $\cos x$ & $\sum_{n=0}^{\infty} \frac{(-1)^n}{(2n)!}x^{2n}$ \\
	\end{tabular}}
    \end{minipage}
};
%------------ Series Solution Header ---------------------
\node[fancytitle, right=10pt] at (box.north west) {Series Solution};
\end{tikzpicture}

%------------ Systems of ODE Content ---------------
\begin{tikzpicture}
\node [mybox] (box){%
    \begin{minipage}{0.3\textwidth}
    \small{
    	\begin{tabular}{lp{4cm} l}
        $\vec{x}' = A\vec{x}$ \\
		\textit{A is diagonalizable} & $\vec{x}(t)=a_{1}e^{\lambda_1 t}\vec{v_1}+\cdots+ a_{n}e^{\lambda_n t}\vec{v_n}$ \\ \hline
        \textit{A is not diagonalizable} & $\vec{x}(t)=a_1e^{\lambda_1 t}\vec{v_1} + a_2e^{\lambda t}(\vec{w} + t\vec{v} )$ \\
        & where $(A - \lambda I)\vec{w} = \vec{v} $\\
        & $\vec{v}$ is an Eigenvector w/ value $\lambda$ \\
        & i.e. $\vec{w}$ is a generalized Eigenvector \\ \hline
        $\vec{x}' = A\vec{x} + \vec{B}$ &Solve $y_h$ \\
        & $\vec{x_1} = e^{\lambda_1t}\vec{v_1}, \vec{x_2} = e^{\lambda_2t}\vec{v_2}$ \\ 			& $\vec{X} = [\vec{x_1},\vec{x_2}]$ \\
        & $\vec{X}\vec{u}'=\vec{B}$ \\
        & $y_p = \vec{X}\vec{u}$ \\
        & $y = y_h + y_p$
	\end{tabular}}
    \end{minipage}
};
%------------ Systems of ODE Header ---------------------
\node[fancytitle, right=10pt] at (box.north west) {Systems};
\end{tikzpicture}

%------------ Modes Content ---------------
\begin{tikzpicture}
\node [mybox] (box){%
    \begin{minipage}{0.3\textwidth}
    \small{
    	\begin{tabular}{lp{4cm} l}
       v enter visual mode \\
       i enter insert mode \\
       r enter replace mode \\
       ESC enter command mode \\
	\end{tabular}}
    \end{minipage}
};
%------------  Modes Header ---------------------
\node[fancytitle, right=10pt] at (box.north west) {Modes};
\end{tikzpicture}
\
%------------ Editing Content ---------------
\begin{tikzpicture}
\node [mybox] (box){%
    \begin{minipage}{0.3\textwidth}
    a enter insert mode and append text \\
    o insert new line below cursor and enter insert mode \\
    x delete character \\
    u undo changes \\
    dd delete line \\
    yy yank or copy line \\
    p paste line below cursor

    \end{minipage}
};
%------------ Editing Header ---------------------
\node[fancytitle, right=10pt] at (box.north west) {Editing};
\end{tikzpicture}
%------------ Content Template ---------------------
\begin{tikzpicture}
\node [mybox] (box){%
    \begin{minipage}{0.3\textwidth}
        \begin{align*}
            \\
            \\
            \\
        \end{align*}
	\end{minipage}
};
%------------ Header Template ---------------------
\node[fancytitle, right=10pt] at (box.north west) {Title template};
\end{tikzpicture}
\\
\\
\\
\\


\end{multicols*}
\end{document}
